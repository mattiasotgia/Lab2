\documentclass[fleqn,varvw,11pt,tightenlines]{memo}

\usepackage[utf8]{inputenc}
\usepackage[T1]{fontenc}

\graphicspath{{../../fig/}}
\usepackage[caption=false]{subfig}
\usepackage{enumerate}

\begin{document}

\title{Memo 2: definizione degli strumenti necessari per la misura}

\author{Francesco Polleri}
\email{s5025011@studenti.unige.it}
\author{Mattia Sotgia}
\email{s4942225@studenti.unige.it}

\collaboration{Gruppo A1}
\affiliation{Dipartimento di Fisica, Università degli Studi di Genova, I-16146 Genova, Italia}

\author{Lorenzo Lucentini}
\author{Michele Giorgi}
\collaboration{Gruppo C6}
\affiliation{Dipartimento di Fisica, Università degli Studi di Genova, I-16146 Genova, Italia}

\revised{\today}

\begin{abstract}

\end{abstract}
\maketitle

\section{Progettazione del circuito}


\section{Correzione degli errori di misura di $V_H$}

\paragraph{Offset} Facendo un primo test della strumentazione osserveremmo un offset della tensione in uscita dall'amplificatore legata al suo funzionamento. Quindi la relazione \begin{equation}
    V_\text{read} = GV_H = G\frac{iB}{wnq}
\end{equation} diventa \begin{equation}
    V_\text{read} = GV_H + V_\text{offset} = G\frac{iB}{wnq} + V_\text{offset}\label{eq:offset},
\end{equation} che possiamo correggere nel fit dei dati sperimentali e ottenere quindi un valore di offset. 

\paragraph{Effetto longitudinale $V_\text{long}$} Se la sonda è leggermente imprecisa nella costruzione, ovvero se i punti dove misuriamo $V_H$ non sono precisamente ortogonali alla direzione di $\vec{J}$ e della corrente, osserviamo che il campo elettrico che misuriamo attraverso $V_H$ è in realtà diverso da zero in assenza di un campo magnetico $B$, infatti avremo una lettura della corrente che attraversa il circuito. Però osserveremmo che questo contributo si comporta com $B^2$, quindi che non interessa effettivamente la nostra misura in maniera influente. Infatti, eseguendo un fit come \begin{equation}
    V_\text{read} = kB^2 +  GV_H + V_\text{offset} = kB^2 + G\frac{iB}{wnq} + V_\text{offset},\label{eq:B2_dep}
\end{equation} osserviamo che il valore di $k\neq0$, quindi possiamo togliere anche il contributo legato al termine quadratico. 

\subsection{Possibile quantificazione dell'effetto di $V_\text{offset}$ e $V_\text{long}$}
 Rispetto a quanto abbiamo detto sugli effetti che possiamo osservare sulla tensione in uscita considerando $i=0$, possiamo per ogni punto della presa dati anche quantificare il valore di $V_\text{offset}$ e il valore di $V_\text{long}$. 
Infatti, imponendo la corrente $i$ nulla possiamo effettivamente trovare il valore di offset della tensione in uscita dall'amplificatore. In questo modo possiamo quindi ottenere un valore di offset quantitativo reale da confrontare con il valore che otteniamo dal fit di (\ref{eq:offset}). 

La questione vogliamo anche riproporla per quanto concerne il valore di $V_\text{long}$. Qua il discorso invece è più delicato, in quanto non si tratta più di un termine indipendente dalle variabili del plot, ma che invece risulta essere dipendente da $B^2$ come in (\ref{eq:B2_dep}).



\end{document}