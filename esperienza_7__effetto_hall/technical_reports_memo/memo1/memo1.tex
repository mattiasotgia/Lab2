\documentclass[fleqn,varvw]{memo}

\usepackage[utf8]{inputenc}
\usepackage[T1]{fontenc}

\graphicspath{{../fig/}}

\begin{document}

\title{Memo 1: definizione necessità laboratorio, definizione degli strumenti necessari per la misura}

\author{Francesco Polleri}
\email{s5025011@studenti.unige.it}
\author{Mattia Sotgia}
\email{s4942225@studenti.unige.it}

\collaboration{Gruppo A1}
\affiliation{Dipartimento di Fisica, Università degli Studi di Genova, I-16146 Genova, Italia}

\revised{\today}

\begin{abstract}

\end{abstract}
\maketitle

\section{Obiettivo della misurazione}

Verifica dell'effetto Hall, confronto con la previsione teorica del valore previsto di portatori di carica e misura della carica dei singoli portatori.

Definita la densità di corrente e la forza di Lorentz, abbiamo allora che la tensione di Hall si può esprimere come \begin{equation}
    V_H = \frac{iB}{wnq},
\end{equation} con $i$ corrente, $B$ campo magnetico a cui è sottoposta la sonda (fig. !) e $w$ spessore della sonda. Così otteniamo che il valore di $n$ è facilmente ottenibile. 

\section{Definizione PLOT}

Possiamo ugualmente considerare il PLOT di $V_H$ come $V_H(B)$ oppure alternativamente $V_H(i)$, in entrambi i casi abbiamo dei vantaggi e degli svantaggi. Soprattutto considerando la corrente abbiamo lo svantaggio di diver trovare un modo valido di poter misurare la corrente che non influisca in modo significativo con l'apparato sperimentale utilizzato .

\end{document}