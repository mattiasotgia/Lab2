\documentclass[fleqn,varvw,preprintnumbers,citeautoscript]{memo}

\usepackage[utf8]{inputenc}
\usepackage[T1]{fontenc}

\graphicspath{{../../fig/}}
\usepackage[caption=false]{subfig}
\usepackage{enumerate}
\usepackage{soul}


\newcommand\vlong{V_\text{long}}

\begin{document}

\title{Memo 4}

\author{Francesco Polleri}
\email{s5025011@studenti.unige.it}
\author{Mattia Sotgia}
\email{s4942225@studenti.unige.it}

\collaboration{Gruppo A1}
\affiliation{Dipartimento di Fisica, Università degli Studi di Genova, I-16146 Genova, Italia}

\author{Lorenzo Lucentini}
\author{Michele Giorgi}
\collaboration{Gruppo C6}
\affiliation{Dipartimento di Fisica, Università degli Studi di Genova, I-16146 Genova, Italia}

\revised{\today}
\preprint{MEMO/4 (\today)}

\begin{abstract}

\end{abstract}
\maketitle

\section{11 maggio 2022}

\subsection{Generatore di corrente}

Realizzare generatore di corrente, caratterizzazione generatore di corrente: \begin{enumerate}
    \item Resistenze scelte, montare su base;
    \item Applicare una differenza di potenziale [0, 5] V (o \SI{3.3}{\volt}) e verificare che la corrente in uscita con $R_5 = \SI{500}{\ohm}$ è proprio \SI{10}{\milli\ampere}. Scegliere $R_5$ più vicino possibile al valore teorico previsto, meglio maggiore di \SI{500}{\ohm} se non presente uguale.
    \item Raccogliere alcuni coppie di punti $(i,\Delta V)$ e verificare che il fattore di dipendenza lineare sia effettivamente $\simeq1/500$.
    \item Alimentarlo con Arduino (collegando il pin D2 digitale) e verificare il corretto funzionamento.
\end{enumerate}

\subsection{Amplificatore operazionale per strumentazione}

\begin{enumerate}
    \item Resistenze, giù scelta, montare secondo lo schema circuitale;
    \item Trovare $V_\text{offset}$, $G_\text{MC}$ (guadagno di modo comune), e il $G_\text{diff}$ (guadagno differenziale). 
\end{enumerate}

\subsection{Convertitore da [0, 5] Volt a [-12, +12] Volt (RS-232)}

\begin{enumerate}
    \item Collegare il pin \verb-D14- all'input non-invertente del comparatore ad anello aperto;
    \item Verificare quindi che con $\pm Vcc = \pm\SI{12}{\volt}$ l'uscita sarà 12 V con ingresso 0 V e $-12$ V con ingresso $+5$ V.
\end{enumerate}

\section{12 maggio 2022}

\subsection{Verifica del funzionamento del seriale RS-232} 
\begin{enumerate}
    \item Verificare che l'invio di un segnale di dati sul seriale 3 di arduino produca una variazione tra $-12$ e $+12$ volt sull'uscita del convertitore;
    \item Verificare che l'invio di un segnale di corrente ad un certo valore corrisponde all'effettiva produzione di corrente a quel dato valore da parte del generatore di corrente.
\end{enumerate}

\subsection{Presa dati}
\begin{enumerate}
    \item Conoscendo la direzione di scorrimento di corrente $i$, il verso e la direzione di $\vec{B}$, sapendo che la corrente è diretta in verso opposto allo spostamento delle cariche negative $e$, allora abbiamo che \begin{enumerate}
        \item Se il campo è uscente rispetto allo schema circuitale della sonda, allora la tensione che misuriamo sarà una tensione sarà coerente con lo schema, quindi maggiore sull'input non-invertente e minore sull'input invertente. 
        \item Se il campo magnetico è invece entrante allora osserveremo che $V_H$ risulterà essere opposta (In questo caso sarà anche necessario \hl{invertire i collegamenti dell'amplificatore per strumentazione} in quanto arduino non può misurare una differenza di tensione negativa). 
    \end{enumerate}
    Quindi per controllare il segno dei portatori di carica dobbiamo effettivamente verificare che sia rispettata questa terna destrorsa. 
    \item Raccogliere M misure con il campo magnetico polarizzato in un verso, poi cambiarne la direzione e raccogliere altrettante M misure lasciando inalterate tutte le altre variabili del sistema.
\end{enumerate}

\end{document}