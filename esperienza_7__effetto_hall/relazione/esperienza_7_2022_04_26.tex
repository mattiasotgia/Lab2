%% Document created 26 April 2022 automatically 
%% from /Users/massimosotgia/Desktop/uni_at_DIFI/Lab2/setup.py 

%% Copyright (C) Mattia Sotgia et al. 2022
%% Using class revtex4-2.cls
%                                       
%                                       
%██       █████  ██████         ██████  
%██      ██   ██ ██   ██             ██ 
%██      ███████ ██████          █████  
%██      ██   ██ ██   ██        ██      
%███████ ██   ██ ██████  ██     ███████ 
%                                       
%                                       
\documentclass[
    prl,
    % preprint,
    % linenumbers,
    % tightlines,
    reprint, 
    superscriptaddress, 
    altaffilletter, 
    amsmath, 
    amssymb, 
    a4paper,
    varvw]{revtex4-2}

\usepackage[top=1.75cm,bottom=2.5cm,left=1.5cm,right=1.5cm]{geometry}

\usepackage[utf8]{inputenc}
\usepackage[T1]{fontenc}

\usepackage[italian]{babel}

%% revtex4-2 bug-fix
\def\andname{e}
%--------------------
\makeatletter
\let\it@comma@def\active@comma
\makeatother

\usepackage{txfonts}
\usepackage{graphicx}% Include figure files
\graphicspath{{../fig/}}

\usepackage{dcolumn}% Align table columns on decimal point
\usepackage{bm}% bold math
\usepackage[colorlinks, urlcolor=., bookmarks]{hyperref}% add hypertext capabilities
\renewcommand\UrlFont{\color{blue}}

\usepackage{physics}
\usepackage{siunitx}

\usepackage{fancyhdr}
\pagestyle{fancy}
\fancyhf{}
\def\twodigits#1{\ifnum#1<10 0\fi\the#1}

%-----------------------------------------------------------------------------------------------

\usepackage{background}
\SetBgColor{gray}
\SetBgAngle{90}
\SetBgScale{2}
\SetBgVshift{0.27\textwidth}

\usepackage[american resistors]{circuitikz}
\usepackage{listings}
\lstset{
  basicstyle=\fontsize{5}{6}\selectfont\ttfamily,
  % backgroundcolor=\color{white},   % choose the background color
  % basicstyle=\footnotesize,        % the size of the fonts that are used for the code
  breakatwhitespace=false,         % sets if automatic breaks should only happen at whitespace
  breaklines=true,                 % sets automatic line breaking
  captionpos=b,                    % sets the caption-position to bottom
  % commentstyle=\color{mygreen},    % comment style
  deletekeywords={...},            % if you want to delete keywords from the given language
  escapeinside={\%*}{*)},          % if you want to add LaTeX within your code
  % extendedchars=true,              % lets you use non-ASCII characters; for 8-bits encodings only, does not work with UTF-8
  % firstnumber=1000,                % start line enumeration with line 1000
  % frame=single,                    % adds a frame around the code
  % keepspaces=true,                 % keeps spaces in text, useful for keeping indentation of code (possibly needs columns=flexible). 
  % keywordstyle=\color{blue},       % keyword style
  % numbers=left,                    % where to put the line-numbers; possible values are (none, left, right)
  % numbersep=5pt,                   % how far the line-numbers are from the code
  numberstyle=\tiny\color{gray}, % the style that is used for the line-numbers
  % rulecolor=\color{black},         % if not set, the frame-color may be changed on line-breaks within not-black text (e.g. comments (green here))
  showspaces=false,                % show spaces everywhere adding particular underscores; it overrides 'showstringspaces'
  showstringspaces=false,          % underline spaces within strings only
  showtabs=false,                  % show tabs within strings adding particular underscores
  stepnumber=2,                    % the step between two line-numbers. If it's 1, each line will be numbered
  % stringstyle=\color{mymauve},     % string literal style
  tabsize=2,                       % sets default tabsize to 2 spaces
}
\usepackage{soul}


%% Define ref types
\newcommand{\reftab}[1]{Tabella {\ref{#1}}}%
\newcommand{\reffig}[1]{Figura {\ref{#1}}}%
\newcommand{\refeqn}[1]{({\ref{#1}})}%
\newcommand{\ChiSqr}{$\chi^2$\space}
\newcommand{\ChiNdf}{$\chi^2/\text{ndf}$}
\newcommand{\cernroot}{\texttt{root}}
\newcommand{\treSigma}{$3\sigma$}
\newcommand{\stdErr}[1]{$\varepsilon_{#1}$}
\newcommand{\mstdErr}[1]{\varepsilon_{#1}}
%% PAPER ONLY custom Macros

\newenvironment{methods}[1]{\section*{#1}
%\fontfamily{phv}
\fontsize{7.5}{9}\selectfont\label{sec:methods}\noindent}{\par\noindent}

%\usepackage{lcsec}

\usepackage{chemformula}
\sisetup{
    % separate-uncertainty=true,
    % per-mode=symbol,
    round-mode=uncertainty,
    % exponent-mode = scientific
}

\setcounter{secnumdepth}{2}

\fancyfoot[C]{
    \the\year\twodigits\month\twodigits\day/7-\thepage
}
\fancyhead[C]{RELAZIONE DI LABORATORIO \textbf{
    N. 4 % ! <== CAMBIARE (Nessuna rel. -> 00)
    } (\the\year)
}

\begin{document}

\title{Misura della densità di portatori di carica su sonda tramite effetto Hall
}
\thanks{Esperienza n. 7
}

\author{Francesco Polleri}
\email{s5025011@studenti.unige.it}
\author{Mattia Sotgia}
\email{s4942225@studenti.unige.it}
\collaboration{Gruppo A1}
\affiliation{Dipartimento di Fisica, Università degli Studi di Genova, I-16146 Genova, Italia}

\author{Michele Giorgi}
\author{Lorenzo Lucentini}
\collaboration{Gruppo C6}
\affiliation{Dipartimento di Fisica, Università degli Studi di Genova, I-16146 Genova, Italia}


\date{presa dati
    11--12 maggio 2022, consegnata in data 
    \today
}

\begin{abstract}
    L'effetto Hall si verifica quando delle cariche transitano attraverso una corrente $i$ ed un campo magnetico $B$, posti perpendicolari l'uno rispetto all'altro, tali per cui si viene a creare una tensione lungo il terzo asse ortogonale. Questa tensione è direttamente proporzionale a $i$ e $B$, ed inversamente proporzionale alla carica dei portatori e al loro numero $n$.
    Si vuole misurare la densità di portatori di carica di una sonda di bismuto \ch{^{83}Bi} realizzata per deposizione su film. Questa sonda è inserita nel traferro di un circuito magnetico, dove è sottoposta ad un campo $B_t$. La tensione $V_H$ è ortogonale a questi due contributi e può essere misurata direttamente sulla sonda.
\end{abstract}


\maketitle
\thispagestyle{fancy}
% Rimuovere per consegna
\SetBgContents{
    laboratorio2: e7 [non per la consegna] \today % ! Note di versione
}

%%%% CORPO DEL TESTO
%%%% CORPO DEL TESTO

\section{Introduzione}

Il passaggio di corrente attraverso un sottile strato conduttore comporta la presenza di una densità di corrente, attraverso il materiale stesso, $\vec{J}=nq\vec{v}_d$, dove $\vec{v}_d$ è la velocità di drift (o di spostamento) dei portatori di carica, e $n$ indica la densità di portatori di carica che contribuiscono alla corrente, per unità di volume, misurata in \si{\per\cubic\metre}. Sottoponendo la lamina conduttrice ad un campo magnetico sufficientemente uniforme \footnote{Il campo magnetico deve essere posto ortogonalmente alla densità di corrente $\vec{J}$, evitando di dover effettuare anche misurazioni dell'angolo $\alpha$ di orientamento di $\vec{B}$ rispetto a $\vec{J}$.} otteniamo che le cariche saranno quindi sottoposte ad una forza di Lorentz \begin{equation}
    \vec{F}_m = q\vec{v}_d \times \vec{B} = qv_dB\hat{u}_H,\label{eq:lorentz_F_m}
\end{equation} che avrà come risultato diretto lo spostamento dei portatori di carica $q$ nella direzione $\hat{u}_H$. Considerando un materiale conduttore come composto da cariche $q$ (portatori di corrente) immersi in una distribuzione che possiamo considerare uniforme (secondo la fisica classica) di cariche $-q$ \footnote{Stiamo considerando il valore di $q$ in termini assoluti, non ponendoci quindi problemi sul segno dei portatori di un conduttore. Un conduttore è però elettricamente neutro, quindi a dei portatori di carica $q$ devono corrispondere delle cariche $-q$ che bilancino complessivamente elettricamente la carica. }, possiamo allora osservare che dopo un certo tempo \footnote{Definiamo questo tempo \emph{tempo caratteristico}, e osserviamo che è sufficientemente piccolo da poter considerare questo processo praticamente istantaneo} si formerà un campo elettrico $\vec{E}_H = \vec{F}_H/q$ in direzione ortogonale a $\vec{J}$ e $\vec{B}$, che comporterà quindi l'esistenza di una differenza di potenziale agli estremi della lamina definita come \begin{equation}
    V_H = \frac{iB}{nqw},
\end{equation} dove $w$ indica lo spessore della lamina che utilizziamo, che comunque nel nostro apparato sperimentale riusciamo ad avere molto inferiore alle altre dimensioni della sonda utilizzata, e dove $i=JA$ indica il flusso della densità di corrente, ovvero la corrente attraverso una sezione $A$. L'effetto misurato è noto come effetto Hall \cite{Hall_1897}. Misurando la tensione che si viene a creare sulla lamina, possiamo ottenere quindi una misura efficace della densità di portatori in funzione delle altre variabili del nostro setup. 

I portatori di carica di un qualsiasi materiale classico sono elettroni, di carica $e=\SI{1.602176634e-19}{\coulomb}$ \footnote{valore esatto, fonte BIPM, \emph{defining constants}: \url{https://www.bipm.org/en/measurement-units/si-defining-constants}, anche in \cite{Newell_2018}}. 

\section{Metodo sperimentale}
La misura dei portatori di carica viene effettuata utilizzando una sonda di \ch[]{Bi} metallico depositata su una lamina di film isolante, posta in un campo magnetico generato all'interno di un traferro da un elettromagnete che lavora a correnti variabili tra \SI{0.1}{\ampere} e \SI{1.3}{\ampere}. Il campo magnetico così generato risulta essere molto potente e abbastanza stabile. La sonda è poi attraversata ortogonalmente rispetto alla direzione di $\vec{B}_t$ da una corrente $i_s$. 


%\onecolumngrid
\appendix

\setcounter{table}{0}
\renewcommand{\thetable}{A-\Roman{table}}

%\bibliographystyle{plain}
\bibliography{references/IOP.shortcomm.CODATA2017, references/10.2307_2369245}

\end{document}
    
