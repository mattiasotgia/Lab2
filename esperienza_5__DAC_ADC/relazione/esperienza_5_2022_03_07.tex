%% Document created 07 March 2022 automatically 
%% from /Users/massimosotgia/Desktop/uni_at_DIFI/Lab2/setup.py 

%% Copyright (C) Mattia Sotgia et al. 2022
%% Using class revtex4-2.cls
%                                       
%                                       
%██       █████  ██████         ██████  
%██      ██   ██ ██   ██             ██ 
%██      ███████ ██████          █████  
%██      ██   ██ ██   ██        ██      
%███████ ██   ██ ██████  ██     ███████ 
%                                       
%                                       
\documentclass[
    rmp,
    % preprint,
    % linenumbers,
    % tightlines,
    reprint, 
    superscriptaddress, 
    altaffilletter, 
    amsmath, 
    amssymb, 
    a4paper,
    fleqn]{revtex4-2}

\usepackage[top=1.75cm,bottom=2.5cm,left=1.5cm,right=1.5cm]{geometry}

\usepackage[utf8]{inputenc}
\usepackage[T1]{fontenc}

\usepackage[italian]{babel}

%% revtex4-2 bug-fix
\def\andname{e}
%--------------------
\makeatletter
\let\it@comma@def\active@comma
\makeatother

\usepackage{txfonts}
\usepackage{graphicx}% Include figure files
\graphicspath{{../fig/}}

\usepackage{dcolumn}% Align table columns on decimal point
\usepackage{bm}% bold math
\usepackage[colorlinks, urlcolor=., bookmarks]{hyperref}% add hypertext capabilities
\renewcommand\UrlFont{\color{blue}}

\usepackage{physics}
\usepackage{siunitx}

\usepackage{fancyhdr}
\pagestyle{fancy}
\fancyhf{}
\def\twodigits#1{\ifnum#1<10 0\fi\the#1}

%-----------------------------------------------------------------------------------------------

\usepackage{background}
\SetBgColor{gray}
\SetBgAngle{90}
\SetBgScale{2}
\SetBgVshift{0.27\textwidth}

\usepackage[american resistors]{circuitikz}
\usepackage{listings}
\lstset{
  basicstyle=\fontsize{5}{6}\selectfont\ttfamily,
  % backgroundcolor=\color{white},   % choose the background color
  % basicstyle=\footnotesize,        % the size of the fonts that are used for the code
  breakatwhitespace=false,         % sets if automatic breaks should only happen at whitespace
  breaklines=true,                 % sets automatic line breaking
  captionpos=b,                    % sets the caption-position to bottom
  % commentstyle=\color{mygreen},    % comment style
  deletekeywords={...},            % if you want to delete keywords from the given language
  escapeinside={\%*}{*)},          % if you want to add LaTeX within your code
  % extendedchars=true,              % lets you use non-ASCII characters; for 8-bits encodings only, does not work with UTF-8
  % firstnumber=1000,                % start line enumeration with line 1000
  % frame=single,                    % adds a frame around the code
  % keepspaces=true,                 % keeps spaces in text, useful for keeping indentation of code (possibly needs columns=flexible). 
  % keywordstyle=\color{blue},       % keyword style
  % numbers=left,                    % where to put the line-numbers; possible values are (none, left, right)
  % numbersep=5pt,                   % how far the line-numbers are from the code
  numberstyle=\tiny\color{gray}, % the style that is used for the line-numbers
  % rulecolor=\color{black},         % if not set, the frame-color may be changed on line-breaks within not-black text (e.g. comments (green here))
  showspaces=false,                % show spaces everywhere adding particular underscores; it overrides 'showstringspaces'
  showstringspaces=false,          % underline spaces within strings only
  showtabs=false,                  % show tabs within strings adding particular underscores
  stepnumber=2,                    % the step between two line-numbers. If it's 1, each line will be numbered
  % stringstyle=\color{mymauve},     % string literal style
  tabsize=2,                       % sets default tabsize to 2 spaces
}
\usepackage{soul}


%% Define ref types
\newcommand{\reftab}[1]{Tabella {\ref{#1}}}%
\newcommand{\reffig}[1]{Figura {\ref{#1}}}%
\newcommand{\refeqn}[1]{({\ref{#1}})}%
\newcommand{\ChiSqr}{$\chi^2$\space}
\newcommand{\ChiNdf}{$\chi^2/\text{ndf}$}
\newcommand{\cernroot}{\texttt{root}}
\newcommand{\treSigma}{$3\sigma$}
\newcommand{\stdErr}[1]{$\varepsilon_{#1}$}
\newcommand{\mstdErr}[1]{\varepsilon_{#1}}
%% PAPER ONLY custom Macros

\newenvironment{methods}[1]{\section*{#1}
%\fontfamily{phv}
\fontsize{7.5}{9}\selectfont\label{sec:methods}\noindent}{\par\noindent}

%\usepackage{lcsec}


\fancyfoot[C]{
    \the\year\twodigits\month\twodigits\day/5-\thepage
}
\fancyhead[C]{\fontfamily{phv}\fontsize{12}{12}\selectfont RELAZIONE DI LABORATORIO \textbf{
    N. 005 % ! <== CAMBIARE (Nessuna rel. -> 00)
    } (\the\year)
}

\def\thesection{\arabic{section}}
\def\thesubsection{\thesection.\arabic{subsection}}
\def\thesubsubsection{\thesubsection.\arabic{subsubsection}}

\renewcommand\labelitemi{\textendash}
\renewcommand\labelitemii{\textendash}
\renewcommand\labelitemiii{\textendash}
\renewcommand\labelitemiv{\textendash}

\begin{document}

\title{
    Realizzazione di un ADC, note
}
\thanks{Esperienza n. 5
}

\author{Francesco Polleri}
\email{s5025011@studenti.unige.it}
\author{Mattia Sotgia}
\email{s4942225@studenti.unige.it}

\collaboration{Gruppo A1}
\affiliation{Dipartimento di Fisica, Università degli Studi di Genova, I-16146 Genova, Italia}

\date{presa dati
    7 marzo 2022, consegnata in data 
    \today
}

\begin{abstract}

\end{abstract}
\maketitle
\thispagestyle{fancy}
% Rimuovere per consegna
\SetBgContents{
    laboratorio2: e5 [non per la consegna] \today % ! Note di versione
}

%%%% CORPO DEL TESTO
%%%% CORPO DEL TESTO

Mettiamo delle resistenze di valore ideale (in realtà sono leggermente più piccole, ma rimangono sempre tutte uguali $R$ e $2R$) \begin{align*}
    R &= \SI{270}{\kilo\ohm}\\
    2R &= \SI{560}{\kilo\ohm}
\end{align*}
    
\section{Circuito non amplificato}
L'impedenza di uscita totale del nostro circuito è $R = \SI{270}{\kilo\ohm}$, che però diventa confrontabile con il valore di impedenza di ingresso dell'oscilloscopio (circa \SI[]{1}[]{\mega\ohm}), che quindi si \emph{mangia} 1/4 della tensione di ingresso, perchè abbiamo che di fatto l'impedenza in ingresso è confrontabile con l'impedenza in uscita del nostro circuito.

\subsection{Appunti 15 marzo}
Abbiamo notato che le impedenze in uscita del circuito costringevano ad avere tensioni che venivano \emph{mangiate} come se si trovassero in una sorta di partitore di tensione.

\section{Circuito amplificato}

Aggiungendo l'amplificatore, che ha una impedenza di uscita praticamente infinita (maggiore di quella dell'oscilloscopio), otteniamo che questo problema non si pone più, e quindi con un guadagno pari a $G=2$, il valore massimo raggiungibile corrisponde effettivamente a $15/16$ di \SI{10}{\volt}, ovvero \SI{9.3}{\volt}, che è il valore che otteniamo.  

\subsection{Appunti 15 marzo}
Per il circuito amplificato scegliamo quindi di avere delle tensioni in ingresso inferiori, in particolare scegliamo come resistenze valori nominali di \begin{align*}
    R &= \SI{2.70}{\kilo\ohm}\\
    2R &= \SI{5.60}{\kilo\ohm},
\end{align*} con valori reali molto vicini a questi valori nominali entro un errore inferiore al 5\%.

\section{Note sulla progettazione dell'ADC}

Alcuni appunti di cui tenere conto per successivi progetti di circuiti di conversione digitale analogico o analogico digitale.

\begin{itemize}
    \item Ogni volta controllare sul data sheet dello strumento tutte le caratteristiche, gli strumenti vanno generalmente collegati al GND, e necessitano di una tensione di funzionamento, spesso diverse. 
    \item Durante la realizzazione può essere utile procedere pezzo per pezzo, ogni volta testare le uscite
    \item tenere un BNC libero per effettuare misure si tensione ai vari capi degli strumenti, procedere con senso fisico. 
    \item Utilizzare il trigger in modo sensato, utilizzare anche la funzione \verb-single- dell'oscilloscopio per fermare l'immagine al primo trigger.
\end{itemize}
    

%\onecolumngrid
\appendix

\setcounter{table}{0}
\renewcommand{\thetable}{A-\Roman{table}}

\end{document}
    
