\documentclass[italian, a4paper, 10pt, twocolumn]{../../style/lab_unige}
\usepackage[a4paper, margin=1.75cm, footskip=0.25in]{geometry}

\usepackage[utf8]{inputenc}
\usepackage[T1]{fontenc}

\usepackage[italian]{babel}

% \usepackage{biblatex}

\usepackage[bookmarksopen=true, citebordercolor={0 1 0}, linkbordercolor={1 0 0}, urlbordercolor={0 1 1}]{hyperref}
\usepackage[numbered]{bookmark}

\usepackage{graphicx}
\graphicspath{{../fig/}}
\usepackage{array}
\usepackage{tabulary}
\usepackage{booktabs}

% FOUNDAMENTAL
\usepackage{../../style/custom}

\usepackage{physics}

\usepackage{breqn}
\usepackage{cuted}
\usepackage{txfonts}

\usepackage{lipsum}

\usepackage[american resistors]{circuitikz}

%% Define ref types
\newcommand{\reftab}[1]{Tabella {\ref{#1}}}%
\newcommand{\reffig}[1]{Figura {\ref{#1}}}%
\newcommand{\refeqn}[1]{({\ref{#1}})}%
\newcommand{\ChiSqr}{$\chi^2$\space}
\newcommand{\ChiNdf}{$\chi^2/\text{ndf}$}
\newcommand{\cernroot}{\texttt{root}}
\newcommand{\treSigma}{$3\sigma$}
\newcommand{\stdErr}[1]{$\varepsilon_{#1}$}
\newcommand{\mstdErr}[1]{\varepsilon_{#1}}
\newcommand{\gLab}{$g_t~=~(~9.8056~\pm~0.0001~\text{ stat}~)~\text{ m/s}^2$}
%% PAPER ONLY custom Macros


%%
\setlength{\columnsep}{6mm}

\begin{document}
\title{
    %%TITLE_HERE%%
}
\author{
%% altri autori qui %%
Mattia Sotgia\textsuperscript{1}
}

\date{
    \textit{
    \textsuperscript{1}Gruppo A00, Esperienza di laboratorio n. %%NN%% \\
    %\textsuperscript{2}In presenza in laboratorio per la presa dati\\
    Università degli Studi di Genova, Dipartimento di Fisica.\\
    }
    \vspace{2ex}
    (Presa dati
    %%DATE_HERE%%, 15:00– 18:00; Analisi dati 
    <end-date here>)
}
    \twocolumn[
    \begin{@twocolumnfalse}
        \maketitle
        
        \begin{abstract}
            \textit{Obiettivo-- }
        
            \textit{Metodi-- }
        
            \textit{Risultati-- }
        
            \textit{Conclusione-- }
        
        
        \end{abstract}
        \vspace{2em}
    \end{@twocolumnfalse}
    ]

    %%%% CORPO DEL TESTO
    %%%% CORPO DEL TESTO

    \section{Introduzione}
    \label{section:introduction}

    % \section{Strumentazione}
    % \label{section:strument}

    \section{Metodi}
    \label{section:methods}

    % \section{Analisi dati}
    % \label{section:analysis}

    \section{Risultati}
    \label{section:results}

    \section{Conclusione}
    \label{section:conclusion}
    % \subsection{Controlli}
    % \subsection{Possibili errori sistematici}

    \appendix

    \setcounter{table}{0}
    \renewcommand{\thetable}{A\arabic{table}}

    \section{Dati estesi}
    

\end{document}
    
