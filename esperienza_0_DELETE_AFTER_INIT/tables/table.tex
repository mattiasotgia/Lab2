\begin{table}[b!]
    \begin{ruledtabular}
    \caption{Coppie di valori ($I_A$, $V_V$) del circuito con $R_1$.}
    \label{table:R1}
    \begin{tabular}{lcc}
          & Corrente Elettrica [$\mu$A] & Tensione Elettrica [V] \\
          \colrule
        1 & $50.688\pm0.015$            & $0.502\pm0.003$        \\
        2 & $150.36\pm0.03 $            & $1.491\pm0.009$        \\
        3 & $302.39\pm0.09 $            & $2.99 \pm0.023$        \\
        4 & $453.40\pm0.11 $            & $4.49 \pm0.03 $        \\
        5 & $606.57\pm0.13 $            & $6.01 \pm0.04 $        \\
        6 & $807.01\pm0.15 $            & $8.00 \pm0.05 $        \\
        7 & $975.54\pm0.17 $            & $9.68 \pm0.06 $        \\  
    \end{tabular}
  \end{ruledtabular}
\end{table}

\begin{table}[b!]
  \begin{ruledtabular}
    \caption{Coppie di valori ($I_A$, $V_V$) del circuito con $R_2$.}
    \label{table:R2}
    \begin{tabular}{lcc}
          & Corrente Elettrica [$\mu$A] & Tensione Elettrica [V] \\
          \colrule
        1 & $30.395\pm0.011$            & $0.992\pm 0.006$       \\
        2 & $45.692\pm0.014$            & $1.491\pm 0.009$       \\
        3 & $76.296\pm0.019$            & $2.49 \pm 0.02 $       \\
        4 & $153.08\pm0.03 $            & $4.99 \pm 0.03 $       \\
        5 & $199.00\pm0.04 $            & $6.50 \pm 0.04 $       \\
        6 & $244.91\pm0.09 $            & $8.00 \pm 0.05 $       \\
        7 & $300.65\pm0.09 $            & $9.82 \pm 0.06 $       \\   
    \end{tabular}
  \end{ruledtabular}
\end{table}