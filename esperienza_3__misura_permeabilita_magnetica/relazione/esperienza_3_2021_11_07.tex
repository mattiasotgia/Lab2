%% Document created 07 November 2021 automatically 
%% from /Users/massimosotgia/Desktop/uni_at_DIFI/Lab2/setup.py 

%% Copyright (C) Mattia Sotgia et al. 2022
%% Using class revtex4-2.cls
%                                       
%                                       
%██       █████  ██████         ██████  
%██      ██   ██ ██   ██             ██ 
%██      ███████ ██████          █████  
%██      ██   ██ ██   ██        ██      
%███████ ██   ██ ██████  ██     ███████ 
%                                       
%                                       
\documentclass[
    rmp,
    % preprint,
    % linenumbers,
    % tightlines,
    reprint, 
    superscriptaddress, 
    altaffilletter, 
    amsmath, 
    amssymb, 
    a4paper]{revtex4-2}

\usepackage[top=1.75cm,bottom=2.5cm,left=1.5cm,right=1.5cm]{geometry}

\usepackage[utf8]{inputenc}
\usepackage[T1]{fontenc}

\usepackage[italian]{babel}

%% revtex4-2 bug-fix
\def\andname{e}
%--------------------
\makeatletter
\let\it@comma@def\active@comma
\makeatother

\usepackage{txfonts}
\usepackage{graphicx}% Include figure files
\graphicspath{{../fig/}}

\usepackage{dcolumn}% Align table columns on decimal point
\usepackage{bm}% bold math
\usepackage[colorlinks, urlcolor=., bookmarks]{hyperref}% add hypertext capabilities
\renewcommand\UrlFont{\color{blue}}

\usepackage{physics}
\usepackage{siunitx}

\usepackage{fancyhdr}
\pagestyle{fancy}
\fancyhf{}
\def\twodigits#1{\ifnum#1<10 0\fi\the#1}

%-----------------------------------------------------------------------------------------------

\usepackage{background}
\SetBgColor{gray}
\SetBgAngle{90}
\SetBgScale{2}
\SetBgVshift{0.27\textwidth}

\usepackage[american resistors]{circuitikz}
\usepackage{listings}
\lstset{
  basicstyle=\fontsize{5}{6}\selectfont\ttfamily,
  % backgroundcolor=\color{white},   % choose the background color
  % basicstyle=\footnotesize,        % the size of the fonts that are used for the code
  breakatwhitespace=false,         % sets if automatic breaks should only happen at whitespace
  breaklines=true,                 % sets automatic line breaking
  captionpos=b,                    % sets the caption-position to bottom
  % commentstyle=\color{mygreen},    % comment style
  deletekeywords={...},            % if you want to delete keywords from the given language
  escapeinside={\%*}{*)},          % if you want to add LaTeX within your code
  % extendedchars=true,              % lets you use non-ASCII characters; for 8-bits encodings only, does not work with UTF-8
  % firstnumber=1000,                % start line enumeration with line 1000
  % frame=single,                    % adds a frame around the code
  % keepspaces=true,                 % keeps spaces in text, useful for keeping indentation of code (possibly needs columns=flexible). 
  % keywordstyle=\color{blue},       % keyword style
  % numbers=left,                    % where to put the line-numbers; possible values are (none, left, right)
  % numbersep=5pt,                   % how far the line-numbers are from the code
  numberstyle=\tiny\color{gray}, % the style that is used for the line-numbers
  % rulecolor=\color{black},         % if not set, the frame-color may be changed on line-breaks within not-black text (e.g. comments (green here))
  showspaces=false,                % show spaces everywhere adding particular underscores; it overrides 'showstringspaces'
  showstringspaces=false,          % underline spaces within strings only
  showtabs=false,                  % show tabs within strings adding particular underscores
  stepnumber=2,                    % the step between two line-numbers. If it's 1, each line will be numbered
  % stringstyle=\color{mymauve},     % string literal style
  tabsize=2,                       % sets default tabsize to 2 spaces
}
\usepackage{soul}


%% Define ref types
\newcommand{\reftab}[1]{Tabella {\ref{#1}}}%
\newcommand{\reffig}[1]{Figura {\ref{#1}}}%
\newcommand{\refeqn}[1]{({\ref{#1}})}%
\newcommand{\ChiSqr}{$\chi^2$\space}
\newcommand{\ChiNdf}{$\chi^2/\text{ndf}$}
\newcommand{\cernroot}{\texttt{root}}
\newcommand{\treSigma}{$3\sigma$}
\newcommand{\stdErr}[1]{$\varepsilon_{#1}$}
\newcommand{\mstdErr}[1]{\varepsilon_{#1}}
%% PAPER ONLY custom Macros

\newenvironment{methods}[1]{\section*{#1}
%\fontfamily{phv}
\fontsize{7.5}{9}\selectfont\label{sec:methods}\noindent}{\par\noindent}

%\usepackage{lcsec}


\fancyfoot[C]{
    \the\year\twodigits\month\twodigits\day/3-\thepage
}
\fancyhead[C]{\fontfamily{phv}\fontsize{12}{12}\selectfont RELAZIONE DI LABORATORIO \textbf{
    N. 1 % ! <== CAMBIARE (Nessuna rel. -> 00)
    } (\the\year)
}

\begin{document}

\title{Misura della permeabilità magnetica relativa con circuito RLC risonante
}
\thanks{Esperienza n. 3
}

\author{Francesco Polleri}
\email{s5025011@studenti.unige.it}
\author{Mattia Sotgia}
\email{s4942225@studenti.unige.it}

\collaboration{Gruppo A1}
\affiliation{Dipartimento di Fisica, Università degli Studi di Genova, I-16146 Genova, Italia}

\date{presa dati
    9 novembre 2021, analisi dati e relazione in data <data>, consegnata in data
    \today
}

\begin{abstract}
    
\end{abstract}
\maketitle
\thispagestyle{fancy}
% Rimuovere per consegna
\SetBgContents{
    laboratorio2: e3 (non per la consegna) \today % ! Note di versione
}

%%%% CORPO DEL TESTO
%%%% CORPO DEL TESTO

\section*{Introduzione}
Si vuole misurare il valore della permeabilità magnetica di alcuni materiali dati, di cui non conosciamo esatta composizione chimico-fisica ma che possiamo ipotizzare omogenei, lineari e isotropi (LHI) fino al primo grado di approssimazione, avendo a disposizione un rocchetto plastico su cui sono avvolte $N$ spire di rame, nel quale può essere inserito il volume di materiale creato in modo da riempire quasi completamente il rocchetto. Variando il materiale ci aspettiamo di poter misurare i differenti valori della permeabilità magnetica $\mu_R$. 

Poiché i tipi di misure più precisi che siamo capaci a effettuare sono misure di tempo (in termini di periodo e di frequenza) sfruttiamo il circuito risonante RLC per determinare il valore della frequenza di taglio ($\nu_0$), che risulta legata al valore dell'induttanza e della capacità del condensatore. Cambiando il nucleo all'interno del solenoide modifichiamo il valore di $L$ e di conseguenza troveremo un valore differente di $\nu_0$. Dalla misura della frequenza troviamo il valore di $L$, essendo noti i valori delle altre componenti circuitali, e confrontando i diversi valori possiamo trovare $\mu_R$ per ogni materiale.


\begin{methods}{Metodi}
    \textit{Caratterizzazione del circuito RLC---} Il circuito RLC è definito da tre parametri: la frequenza di taglio $\nu_0$, il fattore di qualità $Q$ e il parametro $A$. Analizzando il circuito troviamo infatti che il valore della funzione di trasferimento è dato da \[\big|H[\nu]\big|=\frac{1}{\sqrt{R+\left(\omega L - \frac{1}{\omega C}\right)^2}},\] per cui osserviamo che il suo valore massimo (cioè 1), si ottenga per $\omega=\omega_0$ che è il valore di quella che abbiamo chiamato frequenza di taglio ($\omega_0$ oppure $\nu_0$). Per valori più bassi e più alti di pulsazione e quindi di frequenza, il valore della funzione di trasferimento diminuisce, per cui il circuito si comporta come un filtro passa banda intorno al valore della frequenza di taglio che a seconda dei valori di L e di C del circuito può essere modificata. Allo stesso modo l'equazione della funzione di trasferimento può essere riscritta come \[\big|H[\nu]\big|=\frac{1}{\sqrt{R+Q_{id}^2\left({\nu\over\nu_0}+{\nu_0\over\nu}\right)^2}}\] dove Q è $Q_{id}=\frac{1}{R}\sqrt{\frac{L}{C}}$ (e abbiamo sostituito $\omega$ con $\nu = \omega / 2\pi$ dove $\nu=1 / T$), per cui notiamo che il filtro diventa tanto più selettivo, tanto più diventa grande Q, che viene definito quindi fattore di qualità. Inoltre dobbiamo anche considerare che l'induttanza si comporta in realtà anche come una resistenza, per cui dobbiamo riconsiderare il valore della funzione di trasferimento inserendo questo ulteriore parametro A uguale $A=\left(1+{R_L\over R}\right)^2$ da cui \begin{subequations}\begin{equation}\big|H[\nu]\big|=\frac{1}{\sqrt{A+Q_{id}^2\left({\nu\over\nu_0}+{\nu_0\over\nu}\right)^2}}.\label{eqn:H(A, Q, v0)}\end{equation}
    
    Perciò in base ai valori di resistenza, capacità e induttanza che inseriamo all'interno del circuito possiamo modificare i valori di tali parametri. 

    Analogamente a quanto avviene per il modulo della funzione di trasferimento (in eq. \ref{eqn:H(A, Q, v0)}), possiamo individuare la fase come \begin{equation}\varphi[\nu]=-\arctan(Q\left({\nu\over\nu_0}+{\nu_0\over\nu}\right)\frac{1}{\sqrt{A}})\label{eqn:phi(Q, A, v0)}\end{equation}\end{subequations}
    
    \noindent\textit{Scelta del valore di risonanza---}\label{par:caratterizzazioneRLC} Vogliamo costruire un circuito la cui frequenza di taglio sia circa 3kHz in modo che intorno a questo valore di frequenza il segnale all'interno del circuito non sia disturbato da possibili rumori presenti a frequenze nell'ordine dei 100Hz o da altre interferenze presenti invece quando arriviamo a oltre 20KHz. Un'altra condizione che imponiamo è che il fattore di qualità sia almeno maggiore di 4 in modo che la banda che filtriamo attraverso il circuito sia sufficientemente stretta, ma nello stesso momento vogliamo che questo fattore non sia troppo elevato perch\'e ci\`o renderebbe la banda troppo stretta, rendendo potenzialmente più difficile eseguire un fit dei dati. Inoltre il fattore di qualità è legato ai valori di R, L e C e per le condizioni in cui possiamo operare in laboratorio questi non permettono valori di Q elevati.
    
    \begin{figure}[b]
        \begin{circuitikz}
            \ctikzset{bipoles/oscope/width=1.0}
            \draw (4.5,2)
            node[oscopeshape, fill=gray!20!white](O1){};
            \draw (O1.in 2) to [short, *-] (5.5,1.2) node[ground]{} node[below left]{GND};
            \draw (0,0)
            node[left]{$V_{in}=1$ V} node[below left=4pt]{(BNC in)} 
            to [short, o-] (0.5,0)
            to [L=$10$ mH, i_=$i$] (2,0)
            to [R=$R_L$, resistors/scale=0.4] (2.75,0)
            to [C=$220$ nF] (4,0)
            to [R=$38$ $\Omega$] (4,-2) 
            to [short] (0,-2)
            node[ground]{} node[left=4pt]{GND}
            (3.5,0) to [short, -] node[right]{$V_{out}$} node[below right=4pt]{(BNC out)} (4.5,0)
            to [short, -] (4.5,1.2)
            to [short, -*] (O1.in 1);
            \draw [red, dashed] 
            (-2,2) 
            node[align=left, below right=2pt]{Generatore di\\tensione\\alternata\\(Oscilloscopio)} 
            rectangle (0.5, -3);
        \end{circuitikz}
        \caption{Circuito utilizzato per il filtro passa-banda progettato nell'esperienza, i valori di R, L e C sono i valori nominali riportati sul componente. La resistenza $R_L$ è la resistenza interna all'induttanza, che verifichiamo non essere nulla.}
        \label{fig:circuit}
    \end{figure}

    \noindent\textit{Caratterizzazione dell'induttanza---}\label{par:L} La bobina su cui andiamo a eseguire le misure di $L$ è composta da un rocchetto cilindrico di plastica dura cavo, attorno al quale viene avvolto un filo di rame smaltato, a comporre 900 spire. L'apparato così creato si comporta come un solenoide caratterizzato da \begin{equation}\label{eqn:inductance}L=\frac{\Phi_B}{I}=\mu_0\frac{N^2}{\ell}S\end{equation} con $\Phi[B]=B\cdot NS$ il flusso del campo magnetico di un solenoide in cui scorre corrente $I$, dove consideriamo $n=N/\ell$ ottenendo che il solenoide è caratterizzato da \begin{equation}L=\mu_0 n^2 \ell S.\end{equation}
    Consideriamo il rocchetto di plastica in primo ordine di approssimazione avente permeabilità magnetica relativa pari a 1, valore che non si discosta molto con quanto si può osservare sperimentalmente. 

    Inserendo un materiale all'interno del rocchetto ci aspettiamo una variazione del valore di $L$, dal quale vogliamo ricavare il valore di $\mu_R$ corrispondente al materiale. I materiali risultano avere dimensioni ($a\times a\times h$) uguali a $11.90\times11.90\times68.00$ mm (dimensioni del materiale A) e $12.10\times12.10\times68.00$ mm (secondo materiale, B), e si possono inserire nella cavità del solenoide, ma non risultano coprire tutta la superficie della cavità stessa. Coprono invece tutta la lunghezza del solenoide, eccedendo rispetto al rocchetto, lungo 60.00 mm, di pochi millimetri, per cui ci riserviamo di non considerare effetti di bordo che richiederebbero calcoli non eseguibili sulla base dei dati raccolti.
    
    Dall'equazione \refeqn{eqn:inductance} abbiamo che $L\cdot I=\Phi_B$. Quando inseriamo il materiale il flusso $\Phi_B$ si può ottenere come somma del flusso interno al materiale ed esterno (nello spazio tra il materiale e la bobina). Otteniamo quindi che  \begin{subequations}\begin{equation}L_{eq}\cdot I = \Phi_B^\text{int}+\Phi_B^\text{ext}= \mu_0 n^2\ell\left(a^2\mu_R+\left(S-a^2\right)\right)=\mu_0n^2\ell\left(S+a^2\left(\mu_R-1\right)\right)\end{equation} dove $a^2$ indica la superficie di base del materiale considerato, con il fattore
    \begin{equation}\Phi_B^\text{int}=\mu_0\mu_Rn^2\ell a^2\end{equation}
    che tiene conto della permeabilità magnetica relativa del materiale e il fattore \begin{equation}\Phi_B^\text{ext}=\mu_0n^2 \ell\left(S-a^2\right)\end{equation}\end{subequations} che invece è il flusso fuori dal materiale.

    Da queste considerazioni otteniamo che quindi possiamo ricavare il valore della permeabilità magnetica $\mu_R$ come \begin{equation}\mu_R=\frac{L_eq-\mu_0 n^2 \ell S}{\mu_0 n^2\ell a^2} + 1\end{equation}

    \noindent\textit{Misura di $\nu_0$---}
    
\end{methods}


%\onecolumngrid
\appendix

\setcounter{table}{0}
\renewcommand{\thetable}{A-\Roman{table}}

\end{document}