%% Document created 07 November 2021 automatically 
%% from /Users/massimosotgia/Desktop/uni_at_DIFI/Lab2/setup.py 

%% Copyright (C) Mattia Sotgia et al. 2022
%% Using class revtex4-2.cls
%                                       
%                                       
%██       █████  ██████         ██████  
%██      ██   ██ ██   ██             ██ 
%██      ███████ ██████          █████  
%██      ██   ██ ██   ██        ██      
%███████ ██   ██ ██████  ██     ███████ 
%                                       
%                                       
\documentclass[
    rmp,
    % preprint,
    % linenumbers,
    % tightlines,
    reprint, 
    superscriptaddress, 
    altaffilletter, 
    amsmath, 
    amssymb, 
    a4paper]{revtex4-2}

\usepackage[top=1.75cm,bottom=2.5cm,left=1.5cm,right=1.5cm]{geometry}

\usepackage[utf8]{inputenc}
\usepackage[T1]{fontenc}

\usepackage[italian]{babel}

%% revtex4-2 bug-fix
\def\andname{e}
%--------------------
\makeatletter
\let\it@comma@def\active@comma
\makeatother

\usepackage{txfonts}
\usepackage{graphicx}% Include figure files
\graphicspath{{../fig/}}

\usepackage{dcolumn}% Align table columns on decimal point
\usepackage{bm}% bold math
\usepackage[colorlinks, urlcolor=., bookmarks]{hyperref}% add hypertext capabilities
\renewcommand\UrlFont{\color{blue}}

\usepackage{physics}
\usepackage{siunitx}

\usepackage{fancyhdr}
\pagestyle{fancy}
\fancyhf{}
\def\twodigits#1{\ifnum#1<10 0\fi\the#1}

%-----------------------------------------------------------------------------------------------

\usepackage{background}
\SetBgColor{gray}
\SetBgAngle{90}
\SetBgScale{2}
\SetBgVshift{0.27\textwidth}

\usepackage[american resistors]{circuitikz}
\usepackage{listings}
\lstset{
  basicstyle=\fontsize{5}{6}\selectfont\ttfamily,
  % backgroundcolor=\color{white},   % choose the background color
  % basicstyle=\footnotesize,        % the size of the fonts that are used for the code
  breakatwhitespace=false,         % sets if automatic breaks should only happen at whitespace
  breaklines=true,                 % sets automatic line breaking
  captionpos=b,                    % sets the caption-position to bottom
  % commentstyle=\color{mygreen},    % comment style
  deletekeywords={...},            % if you want to delete keywords from the given language
  escapeinside={\%*}{*)},          % if you want to add LaTeX within your code
  % extendedchars=true,              % lets you use non-ASCII characters; for 8-bits encodings only, does not work with UTF-8
  % firstnumber=1000,                % start line enumeration with line 1000
  % frame=single,                    % adds a frame around the code
  % keepspaces=true,                 % keeps spaces in text, useful for keeping indentation of code (possibly needs columns=flexible). 
  % keywordstyle=\color{blue},       % keyword style
  % numbers=left,                    % where to put the line-numbers; possible values are (none, left, right)
  % numbersep=5pt,                   % how far the line-numbers are from the code
  numberstyle=\tiny\color{gray}, % the style that is used for the line-numbers
  % rulecolor=\color{black},         % if not set, the frame-color may be changed on line-breaks within not-black text (e.g. comments (green here))
  showspaces=false,                % show spaces everywhere adding particular underscores; it overrides 'showstringspaces'
  showstringspaces=false,          % underline spaces within strings only
  showtabs=false,                  % show tabs within strings adding particular underscores
  stepnumber=2,                    % the step between two line-numbers. If it's 1, each line will be numbered
  % stringstyle=\color{mymauve},     % string literal style
  tabsize=2,                       % sets default tabsize to 2 spaces
}
\usepackage{soul}


%% Define ref types
\newcommand{\reftab}[1]{Tabella {\ref{#1}}}%
\newcommand{\reffig}[1]{Figura {\ref{#1}}}%
\newcommand{\refeqn}[1]{({\ref{#1}})}%
\newcommand{\ChiSqr}{$\chi^2$\space}
\newcommand{\ChiNdf}{$\chi^2/\text{ndf}$}
\newcommand{\cernroot}{\texttt{root}}
\newcommand{\treSigma}{$3\sigma$}
\newcommand{\stdErr}[1]{$\varepsilon_{#1}$}
\newcommand{\mstdErr}[1]{\varepsilon_{#1}}
%% PAPER ONLY custom Macros

\newenvironment{methods}[1]{\section*{#1}
%\fontfamily{phv}
\fontsize{7.5}{9}\selectfont\label{sec:methods}\noindent}{\par\noindent}

%\usepackage{lcsec}


\fancyfoot[C]{
    \the\year\twodigits\month\twodigits\day/3-\thepage
}
\fancyhead[C]{\fontfamily{phv}\fontsize{12}{12}\selectfont RELAZIONE DI LABORATORIO \textbf{
    N. 1 % ! <== CAMBIARE (Nessuna rel. -> 00)
    } (\the\year)
}

\begin{document}

\title{Misura della permeabilità magnetica relativa con circuito RLC risonante
}
\thanks{Esperinza n. 3
}

\author{Francesco Polleri}
\email{s5025011@studenti.unige.it}
\author{Mattia Sotgia}
\email{s4942225@studenti.unige.it}

\collaboration{Gruppo A1}
\affiliation{Dipartimento di Fisica, Università degli Studi di Genova, I-16146 Genova, Italia}

\date{presa dati
    9 novembre 2021, analisi dati <date>, relazione in data 
    \today
}

\begin{abstract}
    
\end{abstract}
\maketitle
\thispagestyle{fancy}
% Rimuovere per consegna
\SetBgContents{
    laboratorio2: e3 (non per la consegna) \today % ! Note di versione
}

%%%% CORPO DEL TESTO
%%%% CORPO DEL TESTO

\section{Introduzione}
Si vuole misurare il valore della permeabilità magnetica di alcuni materiali dati, di cui non conosciamo esatta composizione chimico-fisica ma che possiamo ipotizzare omogenei, lineari ed isotropi fino al primo grado di approssimazione, avendo a disposizione un rocchetto plastico su cui sono avvolte $N$ spire di rame, nel quale può essere inserito il volume di materiale creato in modo da riempire quasi completamente il rocchetto. Variando il materiale ci aspettiamo di poter misurare i differenti valori della permeabilità magnetica $\mu_R$. 

Poichè i tipi di misure più precisi che siamo capaci ad effettuare sono misure di tempo (in termini di periodo e di frequenza) sfruttiamo il circuito risonante RLC per determinare il valore della frequenza di taglio ($\nu_0$), che risulta legata al valore dell'induttanza e della capacità del condensatore. Cambiando il nucleo all'interno del solenoide modifichiamo il valore di $L$ e di conseguenza troveremo un valore differente di $\nu_0$. Dalla misura della frequenza troviamo il valore di $L$, essendo noti i valori delle altre componenti circuitali, e confrontando i diversi valori possiamo trovare $\mu_R$ per ogni materiale.

% \begin{figure}
%     \begin{circuitikz}
%         \ctikzset{bipoles/oscope/width=1.0}
%         \draw (4.5,2)
%         node[oscopeshape, fill=gray!20!white](O1){};
%         \draw (O1.in 2) to [short, *-] (5.5,1.2) node[ground]{} node[below left]{GND};
%         \draw (0,0)
%         node[left]{$V_{in}=1$ V} node[below left=4pt]{(BNC in)} 
%         to [short, o-] (0.5,0)
%         to [L=$100$ mH, i_=$i$] (2,0)
%         to [R=$R_L$, resistors/scale=0.4] (2.75,0)
%         to [C=$100$ nF] (4,0)
%         to [R=$180$ $\Omega$] (4,-2) 
%         to [short] (0,-2)
%         node[ground]{} node[left=4pt]{GND}
%         (3.5,0) to [short, -] node[right]{$V_{out}$} node[below right=4pt]{(BNC out)} (4.5,0)
%         to [short, -] (4.5,1.2)
%         to [short, -*] (O1.in 1);
%         \draw [red, dashed] 
%         (-2,2) 
%         node[align=left, below right=2pt]{Generatore di\\tensione\\alternata\\(Oscilloscopio)} 
%         rectangle (0.5, -3);
%     \end{circuitikz}
%     \caption{Circuito utilizzato per il filtro passa-banda progettato nell'esperienza, i valori di R, L e C sono i valori nominali riportati sul componente. La resistenza $R_L$ è la resistenza interna all'induttanza, che verifichiamo non essere nulla.}
%     \label{fig:circuit}
% \end{figure}


%\onecolumngrid
\appendix

\setcounter{table}{0}
\renewcommand{\thetable}{A-\Roman{table}}

\end{document}