\documentclass[
    % rmp,
    % preprint,
    % linenumbers,
    % tightlines,
    reprint, 
    superscriptaddress, 
    altaffilletter, 
    amsmath, 
    amssymb, 
    % nobalancelastpage, 
    a4paper]{revtex4-2}
\usepackage[top=1.75cm,bottom=2.5cm,left=1.5cm,right=1.5cm]{geometry}

\usepackage[utf8]{inputenc}
\usepackage[T1]{fontenc}

\usepackage[italian]{babel}

\def\andname{e}

%% revtex4-2 bug-fix
\makeatletter
\let\it@comma@def\active@comma
\makeatother

\usepackage{txfonts}
\usepackage{graphicx}% Include figure files
\usepackage{dcolumn}% Align table columns on decimal point
\usepackage{bm}% bold math
% \usepackage[
%     % bookmarksopen=true, 
%     % citebordercolor={0 1 0}, 
%     % linkbordercolor={1 0 0}, 
%     % urlbordercolor={0 1 1}
% ]{hyperref}% add hypertext capabilities
% \usepackage[numbered]{bookmark}

\usepackage{physics}

\usepackage{fancyhdr}
\pagestyle{fancy}
\fancyhf{}
\def\twodigits#1{\ifnum#1<10 0\fi\the#1}
\fancyfoot[C]{\the\year\twodigits\month\twodigits\day-\thepage}
\fancyhead[C]{\fontsize{12}{12}\selectfont RELAZIONE DI LABORATORIO \textbf{
    N. 00%%NN%% % ! <== CAMBIARE (Nessuna rel.-> 00)
    } (\the\year)
}

\graphicspath{{../fig/}}

\usepackage{lipsum}
\usepackage{background}
\SetBgColor{gray}
\SetBgAngle{90}
\SetBgScale{2}
\SetBgVshift{0.27\textwidth}

\usepackage[american resistors]{circuitikz}

%% Define ref types
\newcommand{\reftab}[1]{Tabella {\ref{#1}}}%
\newcommand{\reffig}[1]{Figura {\ref{#1}}}%
\newcommand{\refeqn}[1]{({\ref{#1}})}%
\newcommand{\ChiSqr}{$\chi^2$\space}
\newcommand{\ChiNdf}{$\chi^2/\text{ndf}$}
\newcommand{\cernroot}{\texttt{root}}
\newcommand{\treSigma}{$3\sigma$}
\newcommand{\stdErr}[1]{$\varepsilon_{#1}$}
\newcommand{\mstdErr}[1]{\varepsilon_{#1}}
\newcommand{\gLab}{$g_t~=~(~9.8056~\pm~0.0001~\text{ stat}~)~\text{ m/s}^2$}
%% PAPER ONLY custom Macros

\begin{document}

\title{
    %%TITLE_HERE%%
}
\thanks{
    Esperinza n. %%NN%%
}

\author{Francesco Polleri}
\email{s5025011@studenti.unige.it}
\author{Mattia Sotgia}
\altaffiliation{In presenza in laboratorio per la presa dati}
\email{s4942225@studenti.unige.it}

\collaboration{Gruppo A1}
\affiliation{Dipartimento di Fisica, Università degli Studi di Genova, I-16146 Genova, Italia}

\date{Presa dati
    %%DATE_HERE%%, 15:00– 18:00; Analisi dati <date>, Relazione in data 
    \today
}

\begin{abstract}
    \textit{Obiettivo-- }
    
    \textit{Metodi-- }
    
    \textit{Risultati-- }
        
    \textit{Conclusione-- }
        
\end{abstract}
\maketitle
\thispagestyle{fancy}
% Rimuovere per consegna
\SetBgContents{
    Draft:%%TITLE_HERE%% \today % ! Note di versione
}

%%%% CORPO DEL TESTO
%%%% CORPO DEL TESTO

\section{Introduzione}
\label{section:introduction}

% \label{section:strument}
% \section{Strumentazione}

\section{Metodi}
\label{section:methods}

% \section{Analisi dati}
% \label{section:analysis}

\section{Risultati}
\label{section:results}

\section{Conclusione}
\label{section:conclusion}
% \subsection{Controlli}
% \subsection{Possibili errori sistematici}

\appendix

\setcounter{table}{0}
\renewcommand{\thetable}{A-\Roman{table}}

\section{Dati estesi}

\end{document}
    
